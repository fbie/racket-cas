%% LyX 2.0.3 created this file.  For more info, see http://www.lyx.org/.
%% Do not edit unless you really know what you are doing.
\documentclass[english]{scrartcl}

\usepackage{babel}
\usepackage[T1]{fontenc}
\usepackage[utf8]{inputenc}

\usepackage{fancyhdr}
\pagestyle{fancy}

\usepackage{graphicx}

\usepackage[unicode=true,
 bookmarks=true,bookmarksnumbered=false,bookmarksopen=false,
 breaklinks=false,pdfborder={0 0 1},backref=false,colorlinks=true]
 {hyperref}
\hypersetup{pdftitle={Implementing Quantified Expressions for OpenJML},
  pdfauthor={Florian Biermann}, pdfsubject={SPLG Project Spring E2012},
  pdfkeywords={Racket, Gradual typing, Lock-free concurrency}}

\usepackage{color}
 
\definecolor{dkgreen}{rgb}{0,0.6,0}
\definecolor{gray}{rgb}{0.5,0.5,0.5}
\definecolor{mauve}{rgb}{0.58,0,0.82}

\usepackage{listings}
\lstset{
  language=Lisp,
  basicstyle=\ttfamily\footnotesize,
  numbers=left,
  numberstyle=\tiny\color{gray},
  numbersep=5pt,
  backgroundcolor=\color{white},
  tabsize=2,
  captionpos=b,
  keywordstyle=\color{blue}, commentstyle=\color{dkgreen},
  stringstyle=\color{mauve},
  frame=single,
  breaklines=true,
  breakatwhitespace=true
}

\makeatother

\begin{document}

\subject{SPLG Project Report E2012}

\title{Michael-Scott Queue in Racket}

\author{Florian Biermann -- fbie@itu.dk}

\maketitle

\begin{abstract}
This project report for ``Programming Languages Seminar'', autumn 2012 at ITU,
is concerned with the difficulties of implementing the Michael-Scott Queue in
Racket and Typed Racket. The report shows the implementations in detail, talks
about issues in typing the code and outlines weaknesses in a software CAS approach.

\end{abstract}
\newpage

\tableofcontents{}
\newpage


\section{Introduction and Motivation}

Over the course of Programming Languages Seminar, many different advanced
topics in current computer science research have been introduced.
The focus of each technology is varying greatly. For a final project,
it is tempting to try to combine a number of these and see, how they
perform together.

This project is concerned with implementing a Michael-Scott Queue
\cite{michael_simple_1996} in Racket. The topics being combined
here are gradual typing and non-blocking concurrency. This is especially
interesting, as Racket does not feature a dedicated compare-and-swap
(CAS) operation executed in hardware. Therefore, the first step is
to show the implementation of a software CAS in Racket.

The remainder of this report is structured as follows. First, I will
give a detailed description of the implementation, outlining differences
between a Java-like CAS operation and the new CAS-like operation I
implemented for Racket. Further, I will describe how the Michael-Scott
Queue implementation is designed and finally concern about gradually
typing this implementation.

For evaluation, I will show how the Racket implementation of the Michael-Scott
queue performs in comparison. The report ends with a conclusion, discussing
these results and outlining some future work.


\section{Implementation}

In this section, I will give a detailed overview of the implementation
of each data structure I implemented over the course of the project. 


\subsection{Compare-And-Swap}
\label{subsec:cas}

\begin{figure}[t]
  \lstinputlisting[linerange={1-29}]{../racket/atomic-ref.rkt}
  \caption{Untyped code for \emph{atomic}.}
  \label{fig:atomic-untyped}
\end{figure}

To implement a CAS operation without access to hardware instructions,
it is necessary to wrap the data, on which CAS should be performed,
into a container. This container will then be assigned a semaphore
that guards access to the CAS function. When a thread executes CAS,
no other thread may invoke CAS on the same container in parallel.
If it does, its execution will be blocked.

The container is implemented in the polymorphic struct \emph{atomic} as
illustrated in Fig. \ref{fig:atomic-untyped}. The struct holds a mutable field
\emph{ref} that can be manipulated safely through CAS
(Fig. \ref{fig:atomic-untyped}, ll. 11-22).

\emph{atomic} is constructed through a second polymorphic function
\emph{make-atomic} which accepts any type as parameter. \emph{make-atomic}
internally passes a newly created semaphore from the Racket standard library to
the \emph{atomic}, together with the value. The semaphore reference is not
mutable and is therefore consistent for the lifetime of any \emph{atomic},
making CAS save.

Each time CAS is executed on an instance of \emph{atomic}, it calls
an internal method, that performs the actual CAS with \emph{call-with-semaphore},
supplying the semaphore field of the reference. \emph{call-with-semaphore} is a
shorthand for waiting and taking the lock on the semaphore, then execute code,
and finally release the lock again.

Since Racket distinguishes between scheduled Threading and true parallelism,
another kind of semaphore needs to be added. \emph{fsemaphore} is a
parallel-safe semaphore that locks the execution of ``futures''
\footnote{See \url{http://docs.racket-lang.org/reference/futures.html} for
  further details on futures in Racket.}, which is the Racket implementation of
parallel threads. This semaphore is assigned to the field \emph{fsem} on
\emph{atomic} and its locking mechanism encloses the traditional semaphore. This
ensures that \emph{atomic} is safe for single- and multi-processor parallelism.

The only publicly provided functions from this implementation are
\begin{itemize}

\item \emph{make-ref} as a safe constructor for \emph{atomic}

\item standard getter function \emph{atomic-ref} that returns the \emph{ref}
value on the \emph{atomic}

\item CAS as the only operation that allows to manipulate the value of \emph{ref}
on \emph{atomic}

\end{itemize}


\subsection{Michael-Scott Queue}

\begin{figure}[t]
  \lstinputlisting[linerange={37-54}]{../racket/michael-scott-queue.rkt}
  \caption{Untyped implementation of \textsc{Dequeue} of the Michael-Scott
    Queue.}
  \label{fig:msq-dequeue-untyped}
\end{figure}

The implementation of the non-blocking Michael-Scott queue
\cite{michael_simple_1996} is based on the re-vised pseudo code of their
implementation
\footnote{See
  \url{http://www.cs.rochester.edu/research/synchronization/pseudocode/queues.html\#nbq}
  for the re-vised pseudo code of \cite{michael_simple_1996}}, translated into
Racket.

The implementation consists of two structs: \emph{node}, which holds the actual
values and is interlinked with other instances of its type, and \emph{msq}, the
container for the linked list that represents the queue. A constructor function
\emph{make-msq} is used to initialize the linked list correctly,
i.e. instantiate the next pointer of \emph{node} with an \emph{atomic}. By using
Racket as implementation language, the 

As an alteration, \textsc{Dequeue} returns the value of the item that has been
dequeued (see Fig. \ref{fig:msq-dequeue-untyped}, ll.2, 13, 16). This makes the
data structure a more practical for actual usage.

Using a \emph{while}-construct instead of a recursive function makes it easier
to repeat the step in case that the queue is not consistent at any point during
\textsc{Enqueue} or \textsc{Dequeue}. If it was implemented only recursively, for each
inconsistency, that is checked in an \emph{if}-expression without an
\emph{else}-clause, the recursive function would have to be called again. This
would be hard to read and therefore the \emph{while}-loop is the better choice
even in a functional language.

It is clearly visible that, due to the implementation of \emph{atomic}, for wich
CAS is defined, the original implementation of Michael and Scott
\cite{michael_simple_1996} can be translated directly into Racket. This is
especially true for \textsc{Enqueue}, as there is no need to alter the
implementation to make it more useful, opposing \textsc{Dequeue} (see
Fig. \ref{fig:msq-enqueue-untyped} for the implementation).

\begin{figure}[t]
  \lstinputlisting[linerange={24-36}]{../racket/michael-scott-queue.rkt}
  \caption{Untyped implementation of \textsc{Enqueue} of the Michael-Scott
    Queue.}
  \label{fig:msq-enqueue-untyped}
\end{figure}


\subsection{Michael-Scott Queue Variant}

\begin{figure}[t]
  \lstinputlisting[linerange={23-43}]{../racket/michael-scott-queue-var.rkt}
  \caption{Untyped implementation of the Michael-Scott Queue variant by Turon et
    al.}
  \label{fig:msq-var-untyped}
\end{figure}

This variant of the Michael-Scott Queue was introduced by Turon et
al. \cite{turon_logical_2012}. The main difference is, that there is no tail
pointer that can lag behind. The tail is determined by recursing to the end of
the queue. This is a valid approach, as at no time the \emph{next} reference of
a node is altered after it was changed from \emph{void} to an actual succeeding
node, even if it got dequeued. This is elaborated further in
\cite{turon_logical_2012}.

The implementation is purely functional, as the pseudo-code of the variant
itself suggests, not using any kind of while-construct but only recursive
calls. This is possible as there is no tail in this variant which could be
lagging behind, making updating the tail a redundant task. The implementation
is illustrated in Fig \ref{fig:msq-var-untyped}.

While this is easier to code, the performance cost for finding the tail in each
step is great in comparison to the original proposed implementation.


\subsection{Typed Version}
\label{subsec:typed}

\begin{figure}[t]
  \lstinputlisting[linerange={26-43}]{../typed-racket/typed-michael-scott-queue.rkt}
  \caption{Typed implementation of \textsc{Enqueue}.}
  \label{fig:typed-enqueue}
\end{figure}

For the sake of staying within the scope of this project, only the re-vised Michael-Scott
Queue was translated into typed code.

The first observation is that \emph{while} is not compatible with Typed Racket
and the argument of code readability does not hold anymore. Another issue with
this is however that the current implementation of \textsc{Enqueue} is now using
way more space, in regard to how many attempts to enqueue fail, as some calls to
CAS need to be bound to local variables in order to be able to execute them
before a succeeding recursive call to \emph{try} (see
Fig. \ref{fig:typed-enqueue} for the actual implementation).

The second observation is that, as the \emph{msq} struct is not initialized with
any value but \emph{void}, it cannot be polymorphic but it will hold values of
type \emph{Any}. This could be changed by allowing an initial value for the
queue, which would be set to the tail position of the queue while head remains a
dummy \emph{node}. Also, this makes the typing easier, though lacking possibly
vital information.


\section{Discussion}

\begin{figure}[t]
  \begin{center}
    \includegraphics[bb=0 0 892 528, scale=0.4]{images/futures_visualizer_untyped.png}
  \end{center}
  \caption{Screenshot from the Future Visualizer showing the execution timeline
    for the untyped Michael-Scott Queue implementation. Blue points indicate a
    blocking operation.}
  \label{fig:untyped-visualizer}
\end{figure}

According to Swaine et al. \cite{swaine_seeing_2012}, programms written in Typed
Racket have advantages over programms written in plain Racket, when it comes to
parallelism. When a parallel program is compiled, the Racket compiler determines
the safety of the code for parallel execution. If the program does not fulfill
all requirements, a so called ``slow path'' will be triggered
\cite{swaine_seeing_2012}, leading eventually towards sequential execution with
the additional overhead paid for managing parallel processes on top.

Swaine et al. claim that typing the program in Typed Racket counters this
behavior, as they also demonstrate in \cite{swaine_seeing_2012}. It makes
therefore sense to measure the amount of parallel work and compare the typed and
the untyped implementation of the Michael-Scott Queue.

Fig. \ref{fig:untyped-visualizer} shows how the untyped implementation does not
perform work in parallel at all. Instead, each thread pauses entirely as soon as
another thread touches the queue.

However, the current typed implementation of the Michael-Scott Queue does not
trigger a ``fast path'' in the Racket compiler. The performance visualizer still
shows 750 ``Blocking'' items for a test with four parallel processes where each
enqueues 250 randomly chosen integers into the same Michael-Scott Queue
instance.

It is likely, that the compiler detects the use of \emph{fsemaphore} objects,
which guard the CAS on \emph{atomic} (see Sec. \ref{subsec:cas}). Since the
Racket compiler regards the main thread not as a parallel thread (even though
it is), it is reasonable to assume that the 1000 occurrences of
\emph{fsemaphore} minus the share of the main thread give the 750 ``Blocking''
items as shown by the visualizer.

It is also possible that this behavior is due to lacking type information, as suggested
in Sec. \ref{subsec:typed}. As Typed Racket and Racket are still undergoing many
changes and are topic of research interest, it was not possible to gain much
information on these corner cases during the duration of this project.

Other tweaks to the code to make the compiler choose a ``fast path'' as
described by SWaine et al. \cite{swaine_seeing_2012} are unfortunately not
adaptable to our case: since they deal with fast fourier transformation, their
approaches target arithmetic operations on and memory allocation of numbers. Since
the Michael-Scott Queue is polymorphic, it is not possible to adapt their
proposed changed to the current implementation.

%\begin{figure}[t]
%  \begin{center}
%    \includegraphics[bb=0 0 892 528, scale=0.4]{images/futures_visualizer_typed.png}
%  \end{center}
%  \caption{Screenshot from the Future Visualizer showing the execution timeline
%    for the typed Michael-Scott Queue implementation.}
%  \label{fig:typed-visualizer}
%\end{figure}


\section{Conclusion and Future Work}

Over the course of this report, we saw that it is very well possible to write a
generally working implementation of the Michael-Scott Queuein Racket as well as
Typed Racket. It became, however, obvious, that true parallelism in Racket is
not as trivial to achieve, because of compiler features that are designed to
stop the programmer from executing possibly error prone code in parallel.

As the reason for this is supposedly the frequent use of semaphores to ensure
correctness on CAS operations, it may be a good idea to add CAS instead as a
built-in function to the Racket compiler. This could then circumvent triggering
``slow paths'' in the execution of parallel programs in Racket and therefore
provide programmers with the possibility to write programmes with fine-grained
parallelism.

It is still possible, that parallel execution is only prevented by too broad
type annotations. Since the project duration is limited, this is not featured
here, but considered for future work.

\bibliographystyle{plain}
\bibliography{SPLG.bib}

\begin{appendix}

  \section{Source Code}
  
  All implemented source code regarding this project, including unit-tests for
  classical and parallel threading, can be found at \url{https://github.com/fbie/racket-cas}.

\end{appendix}

\end{document}



